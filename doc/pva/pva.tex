\documentclass[oneside]{book}

\usepackage{graphicx}
\usepackage[dutch]{babel}
\usepackage[nottoc]{tocbibind}
\usepackage{titlesec}
\usepackage[official]{eurosym}

%\newcommand{\itab}[1]{\hspace{0em}\rlap{#1}}
%\newcommand{\tab}[1]{\hspace{.25\textwidth}\rlap{#1}}
%\newcommand{\stab}[1]{\hspace{.05\textwidth}\rlap{#1}}
%\newcommand{\hl}{\begin{center} \line(1,0){350} \end{center}}
%\newcommand{\hl}{\hspace{\fill}\line(1,0){325}\hspace{\fill}}

\titleformat{\chapter}{\normalfont\Large\bfseries}{\chaptertitlename\ \thechapter{:\ }}{0pt}{\Large}{}
\titlespacing{\chapter}{0pt}{0pt}{2pt}

\title{plan van aanpak\\maze runner}
\author{
	Stephan de Jonge\\
	Stefan de Reuver\\
	Victor Wernet\\
	Nichelle Fleming\\
	Wouter van der Plas
}
\date{\today}
%project naam toevoegen

\begin{document}
\maketitle
\tableofcontents
\pagenumbering{gobble}


\chapter{achtergrond}
de rotterdamse hoogeschool heeft ons gevraag om een robot te bouwen die een doolhof kan doorkruisen.\\
dit moet gebeuren in de snelste tijd. er is niet aangegeven of dat dit project deel van een grooter project.\\
het team, bestaande uit:\\
\begin{itemize}
	\item Wouter van der Plas (Teamleider)
	\item Nichelle Fleming (planner)
	\item Stephan de Jonge (programmeur/bouwer)
	\item Stefan de Reuver (programmeur/bouwer)
	\item Victor Wernet (programmeur/bouwer)
\end{itemize}	
heeft nog weinig ervaaring met het werken met de activity bot maar is zeer gemotieveet.\\


\clearpage
\chapter{projectresultaat}

\clearpage
\chapter{projectactiviteiten}
welken dingen zijn er gedaan
\clearpage
\chapter{projectgrnezen}
wat gaan we wel of niet doen.
\clearpage
\chapter{tussenresultaten}
	de tussen resultaten die worden op gelevert zijn:
\begin{itemize}
	\item een plan van aanpak waarin word beschreven wat we gaan doen tijdens dit project\\
	\item een functioneel ontwerp waarin wij aangeven wat wij gaan maken met een meer techniche visie\\
	\item een prototype van de robot die het dolhof kan door kruisen.\\
	\item en aan het einde van het project een verslag van het manigment
\end{itemize}
\clearpage
\chapter{Kwaliteit}
	De werking van de maze-runner word gemeten in de tijd die hij er overdoet om van A naar B te gaan.\\
	hierbij letten wij aleen op de snelste weg van A naar B. de fisike snellheid kan niet worden aangepas en dus kunnen wij niet sneller dan de snelheid van de robot.
\clearpage
\chapter{Projectorganisatie}
\clearpage
\chapter{Planning}
\clearpage
\chapter{Kosten en Baten}
de kosten die wij maken zijn loon kosten en reis kosten.\\
gemideld verdienen wij 4,-\euro{} \\
en wij reizen prepersoon 6,-\euro{} \\ 
\\
wij hebben geen kosten aan de materiaalen omdat deze worden verzorgt door opdrachtgever.\\
\\
er word verwacht dat wij ((x)) uur aan dit project word besteed.\\
dus zijn de kosten ((y)). aleen besteen aan man uren.\\
\\
wat wij daar voor gaan leveren is een werken prototype van de maze-runner.
\clearpage

\end{document}