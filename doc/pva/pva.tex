\documentclass[oneside]{book}

\usepackage{graphicx}
\usepackage[dutch]{babel}
\usepackage[nottoc]{tocbibind}
\usepackage{titlesec}

%\newcommand{\itab}[1]{\hspace{0em}\rlap{#1}}
%\newcommand{\tab}[1]{\hspace{.25\textwidth}\rlap{#1}}
%\newcommand{\stab}[1]{\hspace{.05\textwidth}\rlap{#1}}
%\newcommand{\hl}{\begin{center} \line(1,0){350} \end{center}}
%\newcommand{\hl}{\hspace{\fill}\line(1,0){325}\hspace{\fill}}

\titleformat{\chapter}{\normalfont\Large\bfseries}{\chaptertitlename\ \thechapter{:\ }}{0pt}{\Large}{}
\titlespacing{\chapter}{0pt}{0pt}{2pt}

\title{plan van aanpak\\maze runner}
\author{
	Stephan de Jonge\\
	Stefan de Reuver\\
	Victor Wernet\\
	Nichelle Fleming\\
	Wouter van der Plas
}
\date{\today}
%project naam toevoegen

\begin{document}
\maketitle
\tableofcontents
\pagenumbering{gobble}


\chapter{achtergrond}
de rotterdamse hoogeschool heeft ons gevraag om een robot te bouwen die een doolhof kan doorkruisen.\\
dit moet gebeuren in de snelste tijd. en er is niet aangegeven of dat dit project deel van een grooter project.\\
het team, bestaande uit:\\
\begin{itemize}
	\item Wouter van der Plas (Teamleider)
	\item Nichelle Fleming (planner)
	\item Stephan de Jonge (programmeur/bouwer)
	\item Stefan de Reuver (programmeur/bouwer)
	\item Victor Wernet (programmeur/bouwer)
\end{itemize}	
heeft nog weinig ervaaring met het werken met de activity bot maar is zeer gemotieveet.\\


\clearpage
\chapter{projectresultaat}
\clearpage
\chapter{projectactiviteiten}
welken dingen zijn er gedaan
\clearpage
\chapter{projectgrnezen}
wat gaan we wel of niet doen.
\clearpage
\chapter{tussen resultaten}
\clearpage
\chapter{Kwaliteit}
\clearpage
\chapter{Projectorganisatie}
\clearpage
\chapter{Planning}
\clearpage
\chapter{Kosten en Baten}
\clearpage

\end{document}