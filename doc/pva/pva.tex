\documentclass[oneside]{book}
\pagestyle{plain}

\usepackage{graphicx}
\usepackage[dutch]{babel}
\usepackage[nottoc]{tocbibind}
\usepackage{titlesec}
\usepackage[official]{eurosym}

%\newcommand{\itab}[1]{\hspace{0em}\rlap{#1}}
%\newcommand{\tab}[1]{\hspace{.25\textwidth}\rlap{#1}}
%\newcommand{\stab}[1]{\hspace{.05\textwidth}\rlap{#1}}
%\newcommand{\hl}{\begin{center} \line(1,0){350} \end{center}}
%\newcommand{\hl}{\hspace{\fill}\line(1,0){325}\hspace{\fill}}

\titleformat{\chapter}{\normalfont\Large\bfseries}{\chaptertitlename\ \thechapter{:\ }}{0pt}{\Large}{}
\titlespacing{\chapter}{0pt}{0pt}{2pt}

\title{plan van aanpak\\
maze-runner\\
\normalsize voor de Hoogeschool Rotterdam
}
\author{
	Stephan de Jonge (0901653@hr.nl)\\
	Stefan de Reuver (0890032@hr.nl)\\
	Victor Wernet (0903258@hr.nl)\\
	Nichelle Fleming (0902117@hr.nl)\\
	Wouter van der Plas (0898649@hr.nl)
}
\date{\today, Rotterdam}
%project naam toevoegen

\begin{document}
\maketitle
\tableofcontents


\chapter{achtergrond}

%stakeholders elvira(FIXME), van waalwijk 
De Rotterdamse hoogeschool heeft ons in een verbale opdracht gegeven om een robot te bouwen die een doolhof kan doorkruisen.\\
Later werd hier aan toegevoegt dat er ook een robot moet komen die tot een afgrond moet kunnen rijden.\\
dit moet gebeuren in de snelste tijd. er is niet aangegeven of dat dit project deel van een grooter project.\\
het team, bestaande uit:\\
\begin{itemize}
	\item Wouter van der Plas (Teamleider)
	\item Nichelle Fleming (planner)
	\item Stephan de Jonge (programmeur)
	\item Victor Wernet (programmeur/bouwer)
	\item Stefan de Reuver (bouwer)
\end{itemize}	
Dit is de eerste keer dat deze groep in deze samenstelling werk en dat wij een project van deze schaal doen.\\
\\
De naam komt van de film mazer unner. wij vonden dit passen om dat dit ook over een doolhof gaat.\\
\\
de stakeholders bestaan uit de project groep. en de opdrachtgever: mevrouw van der Ven en mevrouw van muilwijk.

\clearpage
\chapter{projectresultaat}
Het mazerunners team gaat binnen de komende acht weken een werkende robot opleveren(plus documentatie) die rijdend een doolhof doorkruist met behulp van een of meerdere sensoren, al het materiaal dat gebruikt word is verleend door de Hogeschool Rotterdam.\\
ook is er een opdracht voor een robot die zo dicht mogelijk tot een afgrond moeten rijden. deze robot moet in het zelfde tijsaspekt moeten worden afgelevert. en ook voor deze robot geld dat alle benodigd heden worden gesponsort door de hoogeschool\\
\\
Na afronding van dit project leveren wij een werkende robot die een doolhof kan door kruizen en een robot die tot een afgrond kan rijden.
\clearpage
\chapter{projectactiviteiten}
{\large \textbf{ Plan voor het uitvoeren de opdracht opstellen:}}
\begin{itemize}
\item Initiele bespreking van de opdracht (2 uur)
\item Concept plan opstellen voor het uitvoeren van de opdracht (2 uur)
\item Bespreking wijzigingen (2 uur)
\item Plan bijstellen na het doorvoeren van de wijzigingen (2 uur)
\item Definitieve versie planning maken (2 uur)
\end{itemize}

{\large \textbf{Selecteren van de sensoren (onderzoeksopdracht):}}
\begin{itemize}
\item Testen van de infrarood, ultrasound en whiskers (5 uur)
\item Bepalen welke sensoren gebruikt zullen worden (1 uur)
\item Presentatie over de sensoren voorbereiden en geven (1 uur)
\end{itemize}

{\large \textbf{Opstellen van het plan van aanpak:}}
\begin{itemize}
\item Verzamelen en bestuderen van informatie (3 uur)
\item Gesprekken met de opdrachtgever en andere deskundigen (2 uur)
\item Concept plan van aanpak maken (3 uur)
\item Individueel feedback geven op het plan van aanpak van een andere projectgroep  (2 uur)
\item Bespreking van ontvangen feedbackformulieren (2 uur)
\item Definitieve versie plan van aanpak maken (4 uur)
\end{itemize}

{\large \textbf{Monteren van de sensoren:}}
\begin{itemize}
\item Plan montage opstellen (3 uur)
\item Montage 1ste concept (2 uur)
\item Montage bijstellen tot definitieve versie na testen (8 uur)
\end{itemize}

{\large \textbf{Programmeren van de code voor de montage:}}
\begin{itemize}
\item Plan sample code opstellen (2 uur)
\item Meerdere sample codes programmeren  voor montage 1ste concept (4 uur)
\item Code bijwerken tot definitieve versie na testen (8 uur)
\end{itemize}

\clearpage

{\large \textbf{Maken van het groepsdossier (rapporteren):}}
\begin{itemize}
\item Concept managementsamenvatting maken (3 uur)
\item Definitieve versie managementsamenvatting maken (4 uur)
\item Proefpresentatie voorbereiden en geven (4 uur)
\item Eindpresentatie voorbereiden en geven (3 uur)
\item Concept plan van aanpak, feedbackformulier ingevuld door medestudenten, feedbackformulier docent, definitief plan van aanpak en managementsamenvatting tot een groepsdossier samenstellen en inleveren (2 uur)
\end{itemize}

{\large \textbf{Maken van het individueel dossier:}}
\begin{itemize}
\item Een individueel procesverslag schrijven (3 uur)
\item Feedbackformulier proefpresentatie docent, samenvatting van de feedback van medestudenten, losse feedbackformulieren van medestudenten en \item individueel geschreven onderdeel managementsamenvatting samenstellen tot een individueel dossier en inleveren (2 uur)
\end{itemize}

{\large \textbf{Maken projectdossier (samenwerken):}}
\begin{itemize}
\item Individueel onderdeel managementsamenvatting schrijven (3 uur)
\item Reflectieverslag naar aanleiding van de thema’s “het geven van je mening”, “assertiviteit” en “het geven en ontvangen van feedback” schrijven (3 uur)
\item Verslagen samenstellen tot een projectdossier (1 uur)
\end{itemize}

\begin{table}[h]
\begin{tabular}{|l|l|l|l|l}
\cline{1-4}
code & Omschrijving & Uren & \begin{tabular}[c]{@{}l@{}}Kan\\ pas ná\end{tabular} &  \\ \cline{1-4}
A & \begin{tabular}[c]{@{}l@{}}Plan\\ voor het uitvoeren de opdracht opstellen\end{tabular} & 10 & - &  \\ \cline{1-4}
B & \begin{tabular}[c]{@{}l@{}}Selecteren\\ van de sensoren (onderzoeksopdracht)\end{tabular} & 7 & A &  \\ \cline{1-4}
C & Opstellen,van het plan van aanpak & 16 & A &  \\ \cline{1-4}
D & Monteren,van de sensoren & 13 & B &  \\ \cline{1-4}
E & \begin{tabular}[c]{@{}l@{}}Programmeren\\ van de code voor de montage\end{tabular} & 14 & D &  \\ \cline{1-4}
F & \begin{tabular}[c]{@{}l@{}}Maken\\ van het groepsdossier (rapporteren)\end{tabular} & 16 & \begin{tabular}[c]{@{}l@{}}A, B, C,\\ D en E\end{tabular} &  \\ \cline{1-4}
G & \begin{tabular}[c]{@{}l@{}}Maken\\ van het individueel dossier\end{tabular} & 5 & A,B, C, D en E &  \\ \cline{1-4}
H & \begin{tabular}[c]{@{}l@{}}Maken\\ projectdossier (samenwerken)\end{tabular} & 7 & A,B, C, D en E &  \\ \cline{1-4}
I & \begin{tabular}[c]{@{}l@{}}Challenge\\ uitvoeren\end{tabular} & 2 & \begin{tabular}[c]{@{}l@{}}A, B, C\\ en D\end{tabular} &  \\ \cline{1-4}
\end{tabular}
\end{table}

\clearpage
\chapter{projectgrnezen}
Het project loopt van 20 november tot en met 24 januari hier zitten 2 weken vakantie tussen
Waarin gewerkt kan worden maar waarschijnlijk op een kleiner schaal dan de andere weken.\\
\\
Het maximale budget is vastgesteld op uren, elke week wordt er maximaal 18 uur (per persoon) aan
gewerkt.
Dit cijfer is gebaseerd op het totaal aantal uren.
Dat zijn: 
\begin{itemize}
\item 32 uren voor de techniek binnen het schoolgebouw worden besteed,
\item 25 uren verslaglegging, 4 uren voor de presentatie plus de voorbereiding ervoor,
\item 50 totale uren voor zelf in te delen werkzaamheden(overal aan te besteden binnen het project),
\item 28 uren voor samenwerkings en rapportage vaardigheden met tot slot 1 uur voor de afsluiting en
evaluatie.
\end{itemize}
Alle materialen die voor het project worden gebruikt wordt verleend door de Hogeschool
Rotterdam.
We krijgen aan fysieke goederen: 1 Activitybot van Parallax, kabeltjes, sensoren, weerstanden, een
accu en een set mechano die vrij te gebruiken is.\\
\\
De randvoorwaarden van het project en daarmee ook de grenzen zijn dat de robot zichzelf
autonoom door een doolhof heen kan navigeren met
de verschafte materialen bovendien moet er ook een tweede setup van de robot zijn die zo snel
mogelijk richting de rand van een tafel kan rijden en zichzelf op tijd stopt zodat hij er niet af valt.
Ook moet de documentatie zoals het pva volledig aan de schriftelijke eisen voldoen, alle rapportage
en samenwerkings opdrachten moeten
afgerond zijn en bovendien moet het geheel met een presentatie en een evaluatie zijn afgerond.
Als het laatst genoemde in orde en op tijd ingeleverd/gedaan is, dan is daarmee ook het project
succesvol agerond.
\clearpage
\chapter{tussenresultaten}
	de tussen resultaten die worden op gelevert zijn:
\begin{itemize}
	\item Concept Plan van Aanpak
	\item Definitief Plan van Aanpak
	\item Go/No-go op het Plan van Aanpak
	\item Onderzoeksresultaat sensoren
	\item Ontwerp sensoren Activity bot voor Challenge A
	\item Ontwerp sensoren Activity bot voor Challenge B
	\item Gerealiseerd ontwerp met alle benodigde sensoren aangesloten
	\item Concept code voor Challenge A en B
	\item Test resultaat(rapport) van concept code
	\item Complete versie van Concept code voor Challenge A en B
	\item Test resultaat(rapport) van volledige code
	\item Ontwerp eindpresentatie
	\item Volledige Activity bot mee laten doen aan Challenge A en B
	\item Complete eindpresentatie voordragen samen met de Activity bot
\end{itemize}
\clearpage
\chapter{Kwaliteit}
	Om de kwaliteit van de tussen resultaten en eindresultaat te waarborgen word voor elk resultaat een rapport gemaakt dat vervolgens word gecontroleerd door de projectleider
	 verder worden de resultaten ook besproken in een vergadering die wij wekelijks houden. In deze vergadering word er dan gesproken over de rapport resultaten  de kwaliteit aan te passen van het (tussen) resultaat.
\clearpage
\chapter{Projectorganisatie}
\section*{Algemeen}
Pracktisch word er verantwoording afgelegd bij de beordelende docenten, dit zijn in ons geval: Elvira van der Ven \& Lotte van Muilwijk.
De eindverantwoordelijke voor de communicatie met hen is de projectleider.\\
Er is gekozen voor een communicatieplan waarin de stakeholders beschreven staan.
Dit omdat er niet bijzonder veel stakeholders zijn.
Hierdoor is een omgevingsanalyse te complex voor dit project.
Het communicatieplan kunt vinden onder het kopje stakeholders van dit hoofdstuk.\\
De projectgroep vergadert gemiddeld een keer per week tijdens tussenuren of voor of na school.
Alle vergaderingen bevinden zich binnen de school.\\
De urenverantwoording word gecontroleerd door de projectleider en/of de planner.
Het afrekenen op de urenverantwoording voor projectleden is een taak voor de projectleider.
Het afrekenen op de urenverantwoording van het complete project is aan de opdrachtgever(s).
\section*{Stakeholders}
De volgende stakeholders hebben betrekking op dit project:\\
\\
De Hogeschool Rotterdam, de afdeling CMI en de opleiding TI omdat zij het eindresultaat als
referentie punt kunnen gebruiken om de kennis en vaardigheden van de studenten te kunnen
aantonen aan zichzelf en aan derden.
Ook kunnen zij het resultaat voor marketing en public relations doeleinden gebruiken om nieuwe
aanmelding te stimuleren.\\
\\
De projectondersteunende docenten rekenen wij ook als stakeholders aangezien zij er ook baat bij hebben om een resultaat te zien zodat zij de studenten goed kunnen beordelen.
En zodat zij kunnen zien of er voor de volgende keer aanpassingen aan het project doorgevoerd
moeten worden.\\
\\
Ook zijn de studenten (de projectleden) stakeholders bij dit project aangezien zij tijdens dit project
een hoop relevante kennis voor hun vakgebied kunnen opdoen en omdat ze bij het correct afronden
van dit project maximaal 5 studiepunten kunnen behalen voor hun opleiding.
\clearpage
\section*{Communicatie}
Intern:
De interne communicatie tussen projectleden word geregeld via een *Telegram groeps-chat,
**Whatsapp, mobiele telefonie, vergaderingen en email.\\
\\
 *Telegram, een chat applicatie en service voor Windows, Linux, Mac OS X, Android en IOS apparaten\\
**Whatsapp, een chat applicatie voor Android, IOS, en Windows phone\\
\\
Extern:
De externe communicatie tussen de opdrachtgever, de organisatie en de projectleider gaat via email,
de telefoon, via vergaderingen/bijeenkomsten(mondeling) en schriftelijke opdrachten.

\section*{Archivering}
De archivering van alle relevante informatie binnen het project word geregeld via Github.
Github is een revisie controle systeem/service waarmee men documenten offline kan aanpassen en
later weer kan synchroniseren, bijwerken, aanpassen of verwijderen.
Het werkt door een lokale kopie op te slaan van het project op de computer van een
geauthoriseerde.\\
Deze geauthoriseerde kan de data op zijn eigen computer aanpassen en kan hierna aangeven dat
hij/zij zijn/haar revisie wil samenvoegen met de oorspronkelijke.
Github heeft nog veel meer functies en opties die buiten de scope van dit document vallen.
Voor meer informatie bezoek: https://github.com .
\section*{Het Team}
\textbf{Wouter van der Plas}\\
Functie:Projectleider\\
email: 0898649@hr.nl\\
Mobiel: 0628861310\\
Belbin rollen: Bedrijfsman, vormer\\
Taken: Eindverantwoordelijke voor het  projectresultaat en de documentatie.\\
\\
\textbf{Nichelle Fleming}\\
Functie: Planner\\
email: 0902117@hr.nl\\
Mobiel: 0642503092\\
Belbin rollen: Zorgdrager, plant\\
Taken: Eindverantwoordelijke voor de planning, de documentatie en zij ondersteund de projectleider.\\
\\
\clearpage
\textbf{Stefan de Reuver}\\
Functie: Monteur\\
email: 0890032@hr.nl\\
Mobiel: 0620096064\\
Belbin rollen:\\
Taken: Eindverantwoordelijke voor de montage van de Activitybot.\\
\\
\textbf{Victor Wernet}\\
Functie: Monteur \& Programmeur\\
email: 0903258@hr.nl\\
Mobiel: 0634854013\\
Belbin rollen: Bedrijfsman, zorgdrager\\
Taken: Secundair eindverantwoordelijke voor de montage en programmatuur, ondersteund de monteur en de programmeur.\\
\\
\textbf{Stephan de Jonge}\\
Functie: Programmeur\\
email: 0901653@hr.nl\\
mobiel: 0641782895\\
Belbin rollen: Bedrijfsman, zorgdrager\\
Taken: Eindverantwoordelijke voor de programmatuur van de Activitybot.

\section*{Taakverdeling}
Binnen dit project bestaan de volgende rollen (de omschrijving bevindt zich onder de naam):
\subsection*{Projectleider}
De projectleider is verantwoordelijk voor het uiteindelijke projectresultaat, de planning, de
documentatie, de communicatie met de opdrachtgever en bovendien bewaakt hij/zij het urenbudget
en het fiscale budget.
\subsection*{Planner}
De planner is voornamelijk verantwoordelijk voor de planning.
Hij/zij is samen met de projectleider ook verantwoordelijk voor de documentatie en hij/zij
ondersteund de projectleider in zijn/haar taken.
\subsection*{Monteur}
De monteur is verantwoordelijk voor de montage van het geheel hierna te noemen: Activitybot,
robot of de *“Maze-runner”.
De verantwoordelijkheid betreft de montage van het fysieke deel van de robot, hiermee ook de
betrouwbare werking van de sensoren en de overige electronica.
\subsection*{Programmeur}
De programmeur is verantwoordelijk voor het programmeren van de Activitybot/Maze-runner.
De verantwoordelijkheid betreft het correct functioneren van de geschreven code die de
robot uitvoerd en het correct implementeren van de algoritmes waarmee de robot navigeert.

\section*{Beschikbaarheid}
In principe gaan wij er van uit dat iedereen binnen de projectgroep gemiddeld 18 uur per week
beschikbaar is om aan zijn of haar studielast van in totaal 140 uur te komen.
De kerstvakatie (22-12-14 tot en met 2-1-15) word hier buiten beschouwing gelaten, uiteraard is het
niet ongewoon om verloren uren in deze vakantie in te halen.
Of om gewenste extra uren te werken als het men schikt.

\section*{Bevoegdheden}
De bevoegdheden van de projectleden zijn vastgesteld in een samenwerkingscontract.\\
Zie bijlage “Samenwerkingscontract”.
\clearpage
\chapter{Planning}
week 1\\

Bouwers/programmeurs:\\
\begin{itemize}
\item Onderdelen bestuderen
\item Sensoren testen
\item Onderzoeksopdracht maken
\end{itemize}
	
Planner/projectleider:\\
\begin{itemize}
\item Begin maken aan het plan van aanpak
\end{itemize}

Iedereen:\\
\begin{itemize}
\item Vergaderen over opdracht
\item Concept plan opdracht opstellen
\end{itemize}

week 2\\
Bouwers/programmeurs:\\
\begin{itemize}
\item Experimenteren met de sensoren (infrarood, ultrasound en whiskers)
\item Presentatie geven over de bevindingen en ervaringen met de sensoren
\end{itemize}

Planner/ project leider:\\
\begin{itemize}
\item Concept plan van aanpak maken	
\end{itemize}

Iedereen:\\
\begin{itemize}
\item Vergaderen over wijzigingen plan opdracht
\end{itemize}

week 3\\
Bouwers: \\
\begin{itemize}
\item Plan montage opstellen
\item Montage 1ste concept
\end{itemize}

Programmeurs:\\
\begin{itemize}
\item Plan sample code opstellen
\end{itemize}

Iedereen:\\
\begin{itemize}
\item Definitief plan van aanpak maken
\item Definitief plan opdracht opstellen
\end{itemize}

week 4\\
Bouwers:\\
\begin{itemize}
\item Montage bijstellen
\end{itemize}

Programmeurs:\\
\begin{itemize}
\item Meerdere sample codes voor 1ste montage programmeren
\end{itemize}

Iedereen:\\
\begin{itemize}
\item Werken aan presentatievaardigheden
\item Functionaliteit van de activity-bot in het doolhof testen
\item Vergaderen over de voortgang van het project
\end{itemize}

Week 5\\
Iedereen:\\
\begin{itemize}
\item Montage bijstellen
\item Werken aan de code voor nieuwe functies, debuggen en code opschonen
\item Werken aan een managementsamenvatting
\item Functionaliteit van de activity-bot in het doolhof testen
\item Vergaderen over voortgang (evt. werkzaamheden in de kerstvakantie)
\end{itemize}

week 6\\
Iedereen:\\
\begin{itemize}
\item Montage bijstellen
\item Werken aan een managementsamenvatting
\item Functionaliteit van de activity-bot in de doolhof testen
\item Werken aan de code voor nieuwe functies, debuggen en code opschonen
\item Vergaderen over voortgang 
\end{itemize}

Week 7\\
Iedereen:\\
\begin{itemize}
\item Oefenen met presenteren aan de hand van de feedbackformulieren
\item Montage bijstellen
\item Werken aan de code voor nieuwe functies, debuggen en code opschonen
\item Vergadering over het opleveren van eindproduct 
\item Functionaliteit van de definitieve versie van de code met de activity-bot testen in het doolhof
\end{itemize}

week 8\\
Iedereen:\\
\begin{itemize}
\item Groepsdossier inleveren
\item Individueel dossier inleveren
\item Projectdossier inleveren
\item Eindproduct afronden
\item Eindproduct testen in het doolhof
\item Vergaderen over het uitvoeren van de challenge
\end{itemize}

Week 9\\
Iedereen:\\
\begin{itemize}
\item Eindpresentaties geven
\item Voorbereiden op de challenge door het eindproduct te testen in het doolhof (testrunnen)
\item Challenge uitvoeren
\end{itemize}

\clearpage
\chapter{Kosten en Baten}
De kosten die we zijn tegen gekomen zijn: mensuren, hulpmiddelen, onvoorzien uitgave maar uiteindelijk hebben we een aantal opbrengsten.
Mensuren zijn zeker van belang bij dit project. Elk groepslid zet zichzelf tijdens dit project in om het succesvol af te ronden. Om deze reden is mensuren een van onze grootste kosten. \\

Mensuren tijdens het project kan onderverdeeld worden in 2 onderdelen:
Begeleide projectlesuren: \\
\\
Begeleide instructies techniek gedeelte: \\ 
gedurende 8 weken:  8 * 4  uur  =  32 uur\\
\\
Begeleide instructies rapporteren:  gedurende 8 weken:\\
7 * 2  uur   =  14 uur\\
\\
Begeleide instructies samenwerken:  gedurende 8 weken:\\
7 * 2  uur  =  14  uur\\
                                                      
De in hoofstuk 3 vermelde projectactiviteiten:\\ 
Bestaande uit zelfwerkzaamheid, presentatie eindresultaat, verslaglegging, afsluiting en evaluatie: \\90 uur

In totaal zullen wij 150 uren besteden aan het project.\\

Hulpmiddelen zijn kosten die gemaakt worden om materialen zoals papier, 
hardware en software aan te schaffen. Dit is niet van toepassing op ons, aangezien
we alle benodigde materialen van school gratis krijgen.\\

Onvoorziene uitgaven zijn extra kosten waar niet op waren gerekend. In ons geval
moet het altijd met de opdrachtgever over gesproken worden als die kosten echt
nodig zijn. Dit is niet van toepassing, omdat dat wordt geregeld door de school.\\

Uiteindelijk de opbrengsten is gevormd door alle kosten bij elkaar
samen te tellen. In dit geval is het voor ons niet van toepassing, omdat
we er geen geld voor krijgen, maar de opbrengst wordt in een andere vorm uitbetaalt. 
Wij krijgen door het project goede resultaat af te ronden een voldoende en hebben belangrijke    
kennis   opgedaan, die goed van pas zal komen tijdens het verloop van onze studieloopbaan.
\clearpage
\chapter{risico's}
Bij risico's kan onderscheid worden gemaakt tussen interne
en externe risico's.

\section*{Interne risico's:}
\begin{itemize}
\item Haalbaarheid van het project; Het kan voorkomen dat de gestelde tijd niet genoeg is om het project af te ronden.
\item Iemand die bijvoorbeeld door ziekte niet meer in staat is om het project af te ronden. Dan moeten de overgebleven groepsleden extra taken overnemen. 
\item Projectleden die niet meer met elkaar kunnen of willen samenwerken.
\item Technische risico's; Foute keuze van hardware onderdelen die aan het einde van het project niet meer functioneren. Ook valt hier te denken aan het slecht bouwen van het prototype of het niet goed nadenken hierover.
\item Afhankelijkheid  van andere taken; als de taak 'bouwen van het prototype' door de bouwers niet op tijd gereed is, kan de taak 'software programmeren' door de programmeurs niet beginnen.
\end{itemize}
 

\section*{Externe risico's:}
\begin{itemize}
\item Onvoldoende tijd voor besluitvorming
\item Onduidelijke projectgrenzen. 
\item Veranderingen met betrekking tot de doelstelling kunnen negatieve gevolgen hebben voor het plan van aanpak. Dit brengt veel wijzigingen zich mee om het plan van aanpak relevant te houden aan de nieuwe doelstelling. 
\item De leveringstijd van de materialen kan langer duren dan verwacht aangezien wij met meerdere hardware onderdelen mogen werken bij dit project.   
\end{itemize} 
  
\clearpage

\end{document}