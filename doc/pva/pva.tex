\documentclass[oneside]{book}
\pagestyle{plain}

\usepackage{graphicx}
\usepackage[dutch]{babel}
\usepackage[nottoc]{tocbibind}
\usepackage{titlesec}
\usepackage[official]{eurosym}

%\newcommand{\itab}[1]{\hspace{0em}\rlap{#1}}
%\newcommand{\tab}[1]{\hspace{.25\textwidth}\rlap{#1}}
%\newcommand{\stab}[1]{\hspace{.05\textwidth}\rlap{#1}}
%\newcommand{\hl}{\begin{center} \line(1,0){350} \end{center}}
%\newcommand{\hl}{\hspace{\fill}\line(1,0){325}\hspace{\fill}}

\titleformat{\chapter}{\normalfont\Large\bfseries}{\chaptertitlename\ \thechapter{:\ }}{0pt}{\Large}{}
\titlespacing{\chapter}{0pt}{0pt}{2pt}

\title{plan van aanpak\\maze-runner}
\author{
	Stephan de Jonge (0901653@hr.nl)\\
	Stefan de Reuver (0890032@hr.nl)\\
	Victor Wernet (0903258@hr.nl)\\
	Nichelle Fleming (0902117@hr.nl)\\
	Wouter van der Plas (0898649@hr.nl)
}
\date{\today, Rotterdam}
%project naam toevoegen

\begin{document}
\maketitle
\tableofcontents


\chapter{achtergrond}

%stakeholders elvira(FIXME), van waalwijk 
De Rotterdamse hoogeschool heeft ons in een verbale opdracht gegeven om een robot te bouwen die een doolhof kan doorkruisen.\\
Later werd hier aan toegevoegt dat er ook een robot moet komen die tot een afgrond moet kunnen rijden.\\
dit moet gebeuren in de snelste tijd. er is niet aangegeven of dat dit project deel van een grooter project.\\
het team, bestaande uit:\\
\begin{itemize}
	\item Wouter van der Plas (Teamleider)
	\item Nichelle Fleming (planner)
	\item Stephan de Jonge (programmeur)
	\item Victor Wernet (programmeur/bouwer)
	\item Stefan de Reuver (bouwer)
\end{itemize}	
heeft nog weinig ervaaring met het werken met de activity bot maar ze zijn zeer Gemotiveerd.\\
\\
De naam komt van de film mazerunner. wij vonden dit passen om dat dit ook over een doolhof gaat.\\
\\
de stakeholders bestaan uit de project groep. en de opdrachtgever: mevrouw van der Ven en

\clearpage
\chapter{projectresultaat}
Het mazerunners team gaat binnen de komende acht weken een werkende robot opleveren(plus documentatie) die rijdend een doolhof doorkruist met behulp van een of meerdere sensoren, al het materiaal dat gebruikt word is verleend door de Hogeschool Rotterdam.\\
ook is er een opdracht voor een robot die zo dicht mogelijk tot een afgrond moeten rijden. deze robot moet in het zelfde tijsaspekt moeten worden afgelevert. en ook voor deze robot geld dat alle benodigd heden worden gesponsort door de hoogeschool\\
\\
Na afronding van dit project leveren wij een werkende robot die een doolhof kan door kruizen en een robot die tot een afgrond kan rijden.
\clearpage
\chapter{projectactiviteiten}
welken dingen zijn er gedaan
\clearpage
\chapter{projectgrnezen}
Het project loopt van 20 november tot en met 24 januari hier zitten 2 weken vakantie tussen
Waarin uiteraard wel gewerkt kan worden maar vermoedelijk op een minder grote schaal dan op
school.\\
\\
Het maximale budget is vastgesteld op uren, elke week wordt er maximaal 18 uur (per persoon) aan
gewerkt.
Dit cijfer is gebaseerd op het totaal aantal uren.
Dat zijn: 
\begin{itemize}
\item 32 uren voor de techniek binnen het schoolgebouw worden besteed,
\item 25 uren verslaglegging, 4 uren voor de presentatie plus de voorbereiding ervoor,
\item 50 totale uren voor zelf in te delen werkzaamheden(overal aan te besteden binnen het project),
\item 28 uren voor samenwerkings en rapportage vaardigheden met tot slot 1 uur voor de afsluiting en
evaluatie.
\end{itemize}
Alle materialen die voor het project worden gebruikt wordt verleend door de Hogeschool
Rotterdam.
We krijgen aan fysieke goederen: 1 Activitybot van Parallax, kabeltjes, sensoren, weerstanden, een
accu en een set mechano die vrij te gebruiken is.\\
\\
De randvoorwaarden van het project en daarmee ook de grenzen zijn dat de robot zichzelf
autonoom door een doolhof heen kan navigeren met
de verschafte materialen bovendien moet er ook een tweede setup van de robot zijn die zo snel
mogelijk richting de rand van een tafel kan rijden en zichzelf op tijd stopt zodat hij er niet af valt.
Ook moet de documentatie zoals het pva volledig aan de schriftelijke eisen voldoen, alle rapportage
en samenwerkings opdrachten moeten
afgerond zijn en bovendien moet het geheel met een presentatie en een evaluatie zijn afgerond.
Als het laatst genoemde in orde en op tijd ingeleverd/gedaan is, dan is daarmee ook het project
succesvol agerond.
\clearpage
\chapter{tussenresultaten}
	de tussen resultaten die worden op gelevert zijn:
\begin{itemize}
	\item Een plan van aanpak waarin word beschreven wat we gaan doen tijdens dit project.
	\item Een functioneel ontwerp waarin wij aangeven wat wij gaan maken met Een meer techniche visie.
	\item Een prototype van de robot die het doolhof kan door kruisen.
	\item En aan het einde van het project een verslag van het management.
\end{itemize}
\clearpage
\chapter{Kwaliteit}
	De werking van de maze-runner word gemeten in de tijd die hij er overdoet om van A naar B te gaan.\\
	hierbij letten wij aleen op de snelste weg van A naar B. de fisike snellheid kan niet worden aangepas en dus kunnen wij niet sneller dan de snelheid van de robot.
\clearpage
\chapter{Projectorganisatie}
\section{Algemeen}
Pracktisch word er verantwoording afgelegd bij de beordelende docenten, dit zijn in ons geval: Elvira van der Ven \& Lotte van Muilwijk.
De eindverantwoordelijke voor de communicatie met hen is de projectleider.\\
Er is gekozen voor een communicatieplan waarin de stakeholders beschreven staan.
Dit omdat er niet bijzonder veel stakeholders zijn.
Hierdoor is een omgevingsanalyse te complex voor dit project.
Het communicatieplan kunt vinden onder het kopje stakeholders van dit hoofdstuk.\\
De projectgroep vergadert gemiddeld één keer per week tijdens tussenuren of voor of na school.
Alle vergaderingen bevinden zich binnen de school.\\
De urenverantwoording word gecontroleerd door de projectleider en/of de planner.
Het afrekenen op de urenverantwoording voor projectleden is een taak voor de projectleider.
Het afrekenen op de urenverantwoording van het complete project is aan de opdrachtgever(s).
\section{Stakeholders}
De volgende stakeholders hebben betrekking op dit project:\\
\\
De Hogeschool Rotterdam, de afdeling CMI en de opleiding TI omdat zij het eindresultaat als
referentie punt kunnen gebruiken om de kennis en vaardigheden van de studenten te kunnen
aantonen aan zichzelf en aan derden.
Ook kunnen zij het resultaat voor marketing en public relations doeleinden gebruiken om nieuwe
aanmelding te stimuleren.\\
\\
De projectondersteunende docenten rekenen wij ook als stakeholders aangezien zij er ook baat bij hebben om een resultaat te zien zodat zij de studenten goed kunnen beordelen.
En zodat zij kunnen zien of er voor de volgende keer aanpassingen aan het project doorgevoerd
moeten worden.\\
\\
Ook zijn de studenten (de projectleden) stakeholders bij dit project aangezien zij tijdens dit project
een hoop relevante kennis voor hun vakgebied kunnen opdoen en omdat ze bij het correct afronden
van dit project maximaal 5 studiepunten kunnen behalen voor hun opleiding.
\clearpage
\section{Communicatie}
Intern:
De interne communicatie tussen projectleden word geregeld via een *Telegram groeps-chat,
**Whatsapp, mobiele telefonie, vergaderingen en email.\\
\\
 *Telegram, een chat applicatie en service voor Windows, Linux, Mac OS X, Android en IOS apparaten\\
**Whatsapp, een chat applicatie voor Android, IOS, en Windows phone\\
\\
Extern:
De externe communicatie tussen de opdrachtgever, de organisatie en de projectleider gaat via email,
de telefoon, via vergaderingen/bijeenkomsten(mondeling) en schriftelijke opdrachten.

\section{Archivering}
De archivering van alle relevante informatie binnen het project word geregeld via Github.
Github is een revisie controle systeem/service waarmee men documenten offline kan aanpassen en
later weer kan synchroniseren, bijwerken, aanpassen of verwijderen.
Het werkt door een lokale kopie op te slaan van het project op de computer van een
geauthoriseerde.\\
Deze geauthoriseerde kan de data op zijn eigen computer aanpassen en kan hierna aangeven dat
hij/zij zijn/haar revisie wil samenvoegen met de oorspronkelijke.
Github heeft nog veel meer functies en opties die buiten de scope van dit document vallen.
Voor meer informatie bezoek: https://github.com .
\section{Het Team}
\textbf{Wouter van der Plas}\\
Functie:Projectleider\\
email: 0898649@hr.nl\\
Mobiel: 0628861310\\
Belbin rollen: Bedrijfsman, vormer\\
Taken: Eindverantwoordelijke voor het  projectresultaat en de documentatie.\\
\\
\textbf{Nichelle Fleming}\\
Functie: Planner\\
email: 0902117@hr.nl\\
Mobiel: 0642503092\\
Belbin rollen: Zorgdrager, plant\\
Taken: Eindverantwoordelijke voor de planning, de documentatie en zij ondersteund de projectleider.\\
\\
\textbf{Stefan de Reuver}\\
Functie: Monteur\\
email: 0890032@hr.nl\\
Mobiel: 0620096064\\
Belbin rollen:\\
Taken: Eindverantwoordelijke voor de montage van de Activitybot.\\
\\
\textbf{Victor Wernet}\\
Functie: Monteur \& Programmeur\\
email: 0903258@hr.nl\\
Mobiel: 0634854013\\
Belbin rollen: Bedrijfsman, zorgdrager\\
Taken: Secundair eindverantwoordelijke voor de montage en programmatuur, ondersteund de monteur en de programmeur.\\
\\
\textbf{Stephan de Jonge}\\
Functie: Programmeur\\
email: 0901653@hr.nl\\
mobiel: 0641782895\\
Belbin rollen: Bedrijfsman, zorgdrager\\
Taken: Eindverantwoordelijke voor de programmatuur van de Activitybot.

\section{Taakverdeling}
Binnen dit project bestaan de volgende rollen (de omschrijving bevindt zich onder de naam):
\subsection*{Projectleider}
De projectleider is verantwoordelijk voor het uiteindelijke projectresultaat, de planning, de
documentatie, de communicatie met de opdrachtgever en bovendien bewaakt hij/zij het urenbudget
en het fiscale budget.
\subsection*{Planner}
De planner is voornamelijk verantwoordelijk voor de planning.
Hij/zij is samen met de projectleider ook verantwoordelijk voor de documentatie en hij/zij
ondersteund de projectleider in zijn/haar taken.
\subsection*{Monteur}
De monteur is verantwoordelijk voor de montage van het geheel hierna te noemen: Activitybot,
robot of de *“Maze-runner”.
De verantwoordelijkheid betreft de montage van het fysieke deel van de robot, hiermee ook de
betrouwbare werking van de sensoren en de overige electronica.
\subsection*{Programmeur}
De programmeur is verantwoordelijk voor het programmeren van de Activitybot/Maze-runner.
De verantwoordelijkheid betreft het correct functioneren van de geschreven code/instructies die de
robot uitvoerd en het correct implementeren van de algoritmes waarmee de robot navigeert.

\section{Beschikbaarheid}
In principe gaan wij er van uit dat iedereen binnen de projectgroep gemiddeld 18 uur per week
beschikbaar is om aan zijn of haar studielast van in totaal 140 uur te komen.
De kerstvakatie (22-12-14 tot en met 2-1-15) word hier buiten beschouwing gelaten, uiteraard is het
niet ongewoon om verloren uren in deze vakantie in te halen.
Of om gewenste extra uren te werken als het men schikt.

\section{Bevoegdheden}
De bevoegdheden van de projectleden zijn vastgesteld in een samenwerkingscontract.\\
Zie bijlage “Samenwerkingscontract”.
\clearpage
\chapter{Planning}
\begin{itemize}
\item week 1 
	\begin{itemize}		
	\item \textbf{bouwers/progameurs}
		onderdelen bekijken\\
		sensoren uitproberen\\
		onderzoeksopdracht maken\\
	\item \textbf{planner/projectleider}
		pva opzetten
	\end{itemize}
\item week 2
	\begin{itemize}	
	\item \textbf{bouwers}
		expirimenteren met de sensoren (inflarood ultrasone en whiskers)\\
		presentatie geven over de bevindingen en ervaringen met de sensoren\\
		plan maaken voor het definatiefen aansluiten van de sensoren
	\item \textbf{progameurs}
		van uit het definitief plan een planning maken voor het bepaalen van de fundties die de activitybot moet uitvoeren.
	\item \textbf{planning/ projectleider}
		een conseptueel plan van aanpak maken.
	\end{itemize}
\item week 3
	\begin{itemize}	
	\item \textbf{bouwers/progameurs} 
		testen of dat alle sensoren met elkaar samenwerken.
	\item \textbf{planner/projectleider}
		definitieve versie van het plan van aanpak opleveren
	\end{itemize}
\item week4
	\begin{itemize}	
	\item \textbf{all}
		Vergaderen over de voortgang van het project (1 uur)\\
		werken aan presentatie vaardigheden	
	\item \textbf{progameurs}
		Werken aan de code voor nieuwe functies, debuggen en code opschonen
	\end{itemize}
\item week 5
	\begin{itemize}	
	\item \textbf{all}
		Vergaderen over projectwerkzaamheden in de kert vakantie (1 uur)\\
		functionaliteit van de activitybot in het doolhof testen
	\item \textbf{prjectleider/planner}
		Werken aan de management samenvatting
	\end{itemize}

\item \textbf{KERSTVAKANIE}
\item week 6
	\begin{itemize}	
	\item \textbf{all}
		Vergaderen ver verrichte werkzaamheden in de vakantie
	\item \textbf{projectleider}
		Werken aan management samenvatting
	\item \textbf{bouwers/progameurs}
		Code updaten
		fuctionaliteit van de activitybot in het doolhof testen
	\end{itemize}
\item week 7
	\begin{itemize}	
	\item \textbf{all}
		Oefenen met presenteren van de feedback formulieren\\
		vergaderen over de defenitieve versie van de activitybot
	\item \textbf{progmeurs/bouwers}
		Code opschonen en definitieve versie opleveren van de code
		Fuctionaliteit van de definitieve versie van de code van de activity bot testen in het doolhof
	\end{itemize}
\item week 8
	\begin{itemize}	
	\item \textbf{all}
		Vergaderen over het opleveren van het eindproduct\\
		Eindproduct testen in het doolhof
	\item \textbf{projectleiders}
		Management samenvatting inleveren
	\end{itemize}
\end{itemize}
\clearpage
\chapter{Kosten en Baten}
de kosten die wij maken zijn loon kosten en reis kosten.\\
gemideld verdienen wij 4,-\euro{} \\
en wij reizen per persoon 6,-\euro{} \\ 
\\
wij hebben geen kosten aan de materiaalen omdat deze worden verzorgt door opdrachtgever.\\
\\
er word verwacht dat wij 82 uur aan dit project word besteed.\\
dus zijn de kosten 1344\euro{}. aleen besteen aan man uren.\\
\\
wat wij daar voor gaan leveren is een werken prototype van de maze-runner.
\clearpage
\chapter{risico's}

\clearpage

\end{document}