\documentclass[oneside]{book}
\pagestyle{plain}

\usepackage{graphicx}
\usepackage[dutch]{babel}
\usepackage[nottoc]{tocbibind}
\usepackage{titlesec}
\usepackage[official]{eurosym}

%\newcommand{\itab}[1]{\hspace{0em}\rlap{#1}}
%\newcommand{\tab}[1]{\hspace{.25\textwidth}\rlap{#1}}
%\newcommand{\stab}[1]{\hspace{.05\textwidth}\rlap{#1}}
%\newcommand{\hl}{\begin{center} \line(1,0){350} \end{center}}
%\newcommand{\hl}{\hspace{\fill}\line(1,0){325}\hspace{\fill}}

\titleformat{\chapter}{\normalfont\Large\bfseries}{\chaptertitlename\ \thechapter{:\ }}{0pt}{\Large}{}
\titlespacing{\chapter}{0pt}{0pt}{2pt}

\title{plan van aanpak\\maze-runner}
\author{
	Stephan de Jonge (0901653@hr.nl)\\
	Stefan de Reuver (0890032@hr.nl)\\
	Victor Wernet (0903258@hr.nl)\\
	Nichelle Fleming (0902117@hr.nl)\\
	Wouter van der Plas (0898649@hr.nl)
}
\date{\today, Rotterdam}
%project naam toevoegen

\begin{document}
\maketitle
\tableofcontents


\chapter{achtergrond}

%stakeholders elvira(FIXME), van waalwijk 
De Rotterdamse hoogeschool heeft ons in een verbale opdracht gegeven om een robot te bouwen die een doolhof kan doorkruisen.\\
Later werd hier aan toegevoegt dat er ook een robot moet komen die tot een afgrond moet kunnen rijden.\\
dit moet gebeuren in de snelste tijd. er is niet aangegeven of dat dit project deel van een grooter project.\\
het team, bestaande uit:\\
\begin{itemize}
	\item Wouter van der Plas (Teamleider)
	\item Nichelle Fleming (planner)
	\item Stephan de Jonge (programmeur)
	\item Victor Wernet (programmeur/bouwer)
	\item Stefan de Reuver (bouwer)
\end{itemize}	
heeft nog weinig ervaaring met het werken met de activity bot maar ze zijn zeer Gemotiveerd.\\
\\
De naam komt van de film mazerunner. wij vonden dit passen om dat dit ook over een doolhof gaat.\\
\\
de stakeholders bestaan uit de project groep. en de opdrachtgever: mevrouw van der Ven en

\clearpage
\chapter{projectresultaat}
Het mazerunners team gaat binnen de komende acht weken een werkende robot opleveren(plus documentatie) die rijdend een doolhof doorkruist met behulp van een of meerdere sensoren, al het materiaal dat gebruikt word is verleend door de Hogeschool Rotterdam.\\
ook is er een opdracht voor een robot die zo dicht mogelijk tot een afgrond moeten rijden. deze robot moet in het zelfde tijsaspekt moeten worden afgelevert. en ook voor deze robot geld dat alle benodigd heden worden gesponsort door de hoogeschool\\
\\
Na afronding van dit project leveren wij een werkende robot die een doolhof kan door kruizen en een robot die tot een afgrond kan rijden.
\clearpage
\chapter{projectactiviteiten}
welken dingen zijn er gedaan
\clearpage
\chapter{projectgrnezen}
wel:\\
\begin{itemize}
	\item een robot afleveren die binnen een doolhof van punt A naar punt 
	B kan rijden.
	\item ook kan deze van punt B naar punt A rijden.
\end{itemize}
niet:\\
\begin{itemize}
	\item andere functies, die niet boven worden genoemt, worden niet toevoegen aan deze robot.
\end{itemize}
\clearpage
\chapter{tussenresultaten}
	de tussen resultaten die worden op gelevert zijn:
\begin{itemize}
	\item Een plan van aanpak waarin word beschreven wat we gaan doen tijdens dit project.
	\item Een functioneel ontwerp waarin wij aangeven wat wij gaan maken met Een meer techniche visie.
	\item Een prototype van de robot die het doolhof kan door kruisen.
	\item En aan het einde van het project een verslag van het management.
\end{itemize}
\clearpage
\chapter{Kwaliteit}
	De werking van de maze-runner word gemeten in de tijd die hij er overdoet om van A naar B te gaan.\\
	hierbij letten wij aleen op de snelste weg van A naar B. de fisike snellheid kan niet worden aangepas en dus kunnen wij niet sneller dan de snelheid van de robot.
\clearpage
\chapter{Projectorganisatie}

\clearpage
\chapter{Planning}
\begin{itemize}
\item week 1 
	\begin{itemize}		
	\item \textbf{bouwers/progameurs}
		onderdelen bekijken\\
		sensoren uitproberen\\
		onderzoeksopdracht maken\\
	\item \textbf{planner/projectleider}
		pva opzetten
	\end{itemize}
\item week 2
	\begin{itemize}	
	\item \textbf{bouwers}
		expirimenteren met de sensoren (inflarood ultrasone en whiskers)\\
		presentatie geven over de bevindingen en ervaringen met de sensoren\\
		plan maaken voor het definatiefen aansluiten van de sensoren
	\item \textbf{progameurs}
		van uit het definitief plan een planning maken voor het bepaalen van de fundties die de activitybot moet uitvoeren.
	\item \textbf{planning/ projectleider}
		een conseptueel plan van aanpak maken.
	\end{itemize}
\item week 3
	\begin{itemize}	
	\item \textbf{bouwers/progameurs} 
		testen of dat alle sensoren met elkaar samenwerken.
	\item \textbf{planner/projectleider}
		definitieve versie van het plan van aanpak opleveren
	\end{itemize}
\item week4
	\begin{itemize}	
	\item \textbf{all}
		Vergaderen over de voortgang van het project (1 uur)\\
		werken aan presentatie vaardigheden	
	\item \textbf{progameurs}
		Werken aan de code voor nieuwe functies, debuggen en code opschonen
	\end{itemize}
\item week 5
	\begin{itemize}	
	\item \textbf{all}
		Vergaderen over projectwerkzaamheden in de kert vakantie (1 uur)\\
		functionaliteit van de activitybot in het doolhof testen
	\item \textbf{prjectleider/planner}
		Werken aan de management samenvatting
	\end{itemize}

\item \textbf{KERSTVAKANIE}
\item week 6
	\begin{itemize}	
	\item \textbf{all}
		Vergaderen ver verrichte werkzaamheden in de vakantie
	\item \textbf{projectleider}
		Werken aan management samenvatting
	\item \textbf{bouwers/progameurs}
		Code updaten
		fuctionaliteit van de activitybot in het doolhof testen
	\end{itemize}
\item week 7
	\begin{itemize}	
	\item \textbf{all}
		Oefenen met presenteren van de feedback formulieren\\
		vergaderen over de defenitieve versie van de activitybot
	\item \textbf{progmeurs/bouwers}
		Code opschonen en definitieve versie opleveren van de code
		Fuctionaliteit van de definitieve versie van de code van de activity bot testen in het doolhof
	\end{itemize}
\item week 8
	\begin{itemize}	
	\item \textbf{all}
		Vergaderen over het opleveren van het eindproduct\\
		Eindproduct testen in het doolhof
	\item \textbf{projectleiders}
		Management samenvatting inleveren
	\end{itemize}
\end{itemize}
\clearpage
\chapter{Kosten en Baten}
de kosten die wij maken zijn loon kosten en reis kosten.\\
gemideld verdienen wij 4,-\euro{} \\
en wij reizen per persoon 6,-\euro{} \\ 
\\
wij hebben geen kosten aan de materiaalen omdat deze worden verzorgt door opdrachtgever.\\
\\
er word verwacht dat wij 82 uur aan dit project word besteed.\\
dus zijn de kosten 1344\euro{}. aleen besteen aan man uren.\\
\\
wat wij daar voor gaan leveren is een werken prototype van de maze-runner.
\clearpage
\chapter{risico's}

\clearpage

\end{document}